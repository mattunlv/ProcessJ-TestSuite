\documentclass[pdflatex,11pt,letter]{article}

\usepackage{graphicx}
\begin{document}

\begin{center}
Version 1.0\hfill{}September 30, 2013\\[0.5cm]
{\Huge Imports in ProcessJ}\\[0.5cm]
by Matt B. Pedersen
\end{center}

\section{Introduction}

This document describes the semantics and implementation of the 
{\bf import} statement.

\section{The Import Hierarchy}


For a series of imports like

\noindent
\hspace*{1cm}{\bf import} $std.math$;\\
\hspace*{1cm}{\bf import} $io.files$;\\
\hspace*{1cm}{\bf import} $std.random$;\\
\\
{\tt .. code .. }\\

\noindent
the import hierarchy (and thus the order in which name usage is resolved)
is as follows:\\

\noindent
The file that contains the {\tt .. code ..} has the final symbol table in the 
hierarch, and is thus searched first. If nothing is found there, the 
symbol table representing the import $std.random$ will be searched 
and so forth until finally the $std.math$ symbol table is searched.\\

\noindent
If an import using a * is used, the files in the specified package 
are imported in reverse alphabetical order.\\

\noindent
A symbol table has two links: $parent$ and $importParent$.

\noindent
$importParent$ points to the symbol table of the first imported file
$parent$ point to the next scope which could be another import.\\

\noindent
The way name resolution is done for import hierarchies is slightly 
different than the regular static scoping rules; the reason is that
the look up must be $shallow$: if the main file imports a file $A$ which
imports a file $B$ which declares a type $T$, then this type $T$ should not
be available to the main file without an import of $B$.\\

\noindent
Therefore the resolution goes as follows:\\

\noindent
when looking for a type or a constant look in the symbol table
associated with file; if nothing is found then follow the
link to the $importParent$, but from now on follow the $parent$
links up the chain to avoid doing any deep searches in the 
symbol tables for the imported files of other imports.\\

\noindent
Such is a search is realized quite easily like this:\\

\noindent
\hspace*{1.0cm}// searches the parent chain
\hspace*{1.0cm}{\bf public} $Object$ $get$($String$ $name$) \{\\
\hspace*{1.5cm}  $Object$ $result$ = $entries$.$get$($name$);\\
\hspace*{1.5cm}  {\bf if} ($result$ != {\bf null})\\
\hspace*{2.0cm}    {\bf return} $result$;\\
\hspace*{1.5cm}  {\bf if} ($parent$ == {\bf null}\\
\hspace*{2.0cm}    {\bf return} {\bf null};\\
\hspace*{1.5cm}  {\bf return} $parent$.$get$($name$);\\
\hspace*{1.0cm}\}\\
$ $\\
\hspace*{1.0cm}// searches locally then call get with the importParent\\
\hspace*{1.0cm}// and get then continues up the parent chain.  \\
\hspace*{1.0cm}{\bf public} $Object$ $getIncludeImports$($String$ $name$) \{\\
\hspace*{1.5cm}  $Object$ $result$ = $entries$.$get$($name$);\\
\hspace*{1.5cm}  {\bf if} ($result$ != {\bf null})\\
\hspace*{2.0cm}    {\bf return} $result$;\\
\hspace*{1.5cm}  {\bf if} ($importParent$ == {\bf null})\\
\hspace*{2.0cm}    {\bf return} {\bf null};\\
\hspace*{1.5cm}  {\bf return} $importParent$.$get$($name$);\\
\hspace*{1.0cm}\}\\

\section{Different Import Types}

ProcessJ has two different kinds of import which each come in two different
flavors:

\begin{enumerate}
\item Single file imports like
  \begin{enumerate}
    \item {\bf import} $f$;
    \item {\bf import} $p$.$f$;
    \item {\bf import} $p_1$.$p_2$.$f$;\\
      ...
    \item {\bf import} $p_1$.$p_2$.$...$.$p_n$.$f$;
    \end{enumerate}
    The file import is either a file name by it self, in which case the file
    must be in the same directory as the file that imports it (and in the same
    package!), or a number of package names eventually terminated by a file name.
    the file $p_1$.$p_2$.$f$ must be in the directory {\tt p1/p2} and
    called {\tt f.pj}.\\

    If no local file (relative to where the main file being compiled is), an import
    from the include directory is attempted.
\item Wild card imports like:
  \begin{enumerate}
    \item {\bf import} $p$.*;
    \item {\bf import} $p_1$.$p_2$.*;\\
       ...
    \item {\bf import} $p_1$.$p_2$.$...$.$p_n$.*;
    \end{enumerate}
    Again, for say $p_1$.$p_2$.*, if {\tt p1/p2} does not exist the include directory is searched.

    Wildcard files are imported in reverse alphabetical order, and a
    wild card import is a deep import. That is, if the directory {\tt p1/p2}
    contains other directories, these are also visited and any {\tt .pj}
    file will be imported.
\end{enumerate}


\subsection{Example}

(The filenames followed by nothing have no imports)

\noindent
\underline{\tt Main.pj}\\
{\bf import} Import.A;\\
{\bf import} Import.M;\\
{\bf import} Import.N;\\
$ $\\
\underline{\tt A.pj}\\
{\bf import} Import.B;\\
{\bf import} Import.C;\\
{\bf import} Import.D;\\
$ $\\
\underline{\tt B.pj}\\
{\bf import} Import.E;\\
{\bf import} Import.F;\\
$ $\\
\underline{\tt C.pj}\\
$ $\\
\underline{\tt D.pj}\\
$ $\\
\underline{\tt E.pj}\\
{\bf import} Import.Q;\\
{\bf import} Import.R;\\
$ $\\
\underline{\tt F.pj}\\
{\bf import} Import.T;\\
{\bf import} Import.S;\\
{\bf import} Import.U;\\
$ $\\
\underline{\tt M.pj}\\
$ $\\
\underline{\tt N.pj}\\
$ $\\
\underline{\tt Q.pj}\\
$ $\\
\underline{\tt R.pj}\\
$ $\\
\underline{\tt S.pj}\\
$ $\\
\underline{\tt T.pj}\\
$ $\\
\underline{\tt U.pj}\\

\noindent
With the switch {\tt -sts} on the compiler we get the following import tree:
\newpage
\begin{footnotesize}
\begin{verbatim}
name.........: Global Type Table
parent.......: --//
importParent.: 
|  name.........: Import: /Volumes/Data/Dropbox/ProcessJ/Import/N.pj
|  parent.......: 
|  |  name.........: Import: /Volumes/Data/Dropbox/ProcessJ/Import/M.pj
|  |  parent.......: 
|  |  |  name.........: Import: /Volumes/Data/Dropbox/ProcessJ/Import/A.pj
|  |  |  parent.......: --//
|  |  |  importParent.: 
|  |  |  |  name.........: Import: /Volumes/Data/Dropbox/ProcessJ/Import/D.pj
|  |  |  |  parent.......: 
|  |  |  |  |  name.........: Import: /Volumes/Data/Dropbox/ProcessJ/Import/C.pj
|  |  |  |  |  parent.......: 
|  |  |  |  |  |  name.........: Import: /Volumes/Data/Dropbox/ProcessJ/Import/B.pj
|  |  |  |  |  |  parent.......: --//
|  |  |  |  |  |  importParent.: 
|  |  |  |  |  |  |  name.........: Import: /Volumes/Data/Dropbox/ProcessJ/Import/F.pj
|  |  |  |  |  |  |  parent.......: 
|  |  |  |  |  |  |  |  name.........: Import: /Volumes/Data/Dropbox/ProcessJ/Import/E.pj
|  |  |  |  |  |  |  |  parent.......: --//
|  |  |  |  |  |  |  |  importParent.: 
|  |  |  |  |  |  |  |  |  name.........: Import: /Volumes/Data/Dropbox/ProcessJ/Import/R.pj
|  |  |  |  |  |  |  |  |  parent.......: 
|  |  |  |  |  |  |  |  |  |  name.........: Import: /Volumes/Data/Dropbox/ProcessJ/Import/Q.pj
|  |  |  |  |  |  |  |  |  |  parent.......: --//
|  |  |  |  |  |  |  |  |  |  importParent.: --//
|  |  |  |  |  |  |  |  |  importParent.: --//
|  |  |  |  |  |  |  importParent.: 
|  |  |  |  |  |  |  |  name.........: Import: /Volumes/Data/Dropbox/ProcessJ/Import/U.pj
|  |  |  |  |  |  |  |  parent.......: 
|  |  |  |  |  |  |  |  |  name.........: Import: /Volumes/Data/Dropbox/ProcessJ/Import/S.pj
|  |  |  |  |  |  |  |  |  parent.......: 
|  |  |  |  |  |  |  |  |  |  name.........: Import: /Volumes/Data/Dropbox/ProcessJ/Import/T.pj
|  |  |  |  |  |  |  |  |  |  parent.......: --//
|  |  |  |  |  |  |  |  |  |  importParent.: --//
|  |  |  |  |  |  |  |  |  importParent.: --//
|  |  |  |  |  |  |  |  importParent.: --//
|  |  |  |  |  importParent.: --//
|  |  |  |  importParent.: --//
|  |  importParent.: --//
|  importParent.: --//
\end{verbatim}
\end{footnotesize}

\includegraphics[width=\textwidth]{import-example}
\centerline{Graph version of the import hierarchy.}
\end{document}